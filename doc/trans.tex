\documentclass{article}
\bibliographystyle{unsrt}

% Section
\newcommand{\secref}[1]{Section~\ref{sec.#1}}
\newcommand{\seclabel}[1]{\label{sec.#1}}

% Appendix
\newcommand{\appref}[1]{Appendix~\ref{app.#1}}
\newcommand{\applabel}[1]{\label{app.#1}}

% Table
\newcommand{\tabnum}[1]{\ref{tab.#1}}
\newcommand{\tabref}[1]{Table~\tabnum{#1}}
\newcommand{\tablabel}[1]{\label{tab.#1}}

% Equation
\newcommand{\eqnref}[1]{Equation~\ref{Equations.#1}}
\newcommand{\eqnlabel}[1]{\label{Equations.#1}}

% Figure
\newcommand{\figlabel}[1]{\label{Figures.#1}}
\newcommand{\figref}[1]{Figure~\ref{Figures.#1}}
\newcommand{\subfigref}[2]{Figure~\ref{Figures.#1}(#2)}

% Algorithm
\newcommand{\alglabel}[1]{\label{Algorithms.#1}}
\newcommand{\algref}[1]{Algorithm~\ref{Algorithms.#1}}

% \easyfig{filename.png}{width=...}{label}{caption}
\newcommand{\easyfig}[4]{
\begin{figure}
\includegraphics[#2]{#1}
\caption{ \figlabel{#3} #4}
\end{figure}}

\newcommand{\includedot}[2]{
\input{#1.tex}
\includegraphics[#2]{#1.ps}
}

% \dotfig{filenameprefix}{width=...}{label}{caption}
\newcommand{\dotfig}[4]{
\includedot{#1}{#2}
\begin{figure}
\caption{ \figlabel{#3} #4}
\end{figure}}

% ToDo
\newcommand{\needfig}[1]{{\bf Need figure: } #1 }
\newcommand{\needfigref}[1]{Figure~??? [#1] }

\newcommand{\needcite}[1]{[CITE #1]}
\newcommand{\todo}[1]{[TODO: #1]}

% Packages

\usepackage[boxed,noend]{algorithm2e}
\usepackage{graphicx}
\usepackage{psfrag}
\usepackage{url}
\usepackage{amsmath}
\usepackage{amssymb}
\usepackage{color} 
\usepackage{ifthen}
\usepackage{rotating}

\newcommand\semiring{K}
\newcommand\srset{\mathbb{K}}
\newcommand\srplus{\oplus}
\newcommand\srtimes{\otimes}
\newcommand\srplusid{\bar{0}}
\newcommand\srtimesid{\bar{1}}
\newcommand\srsum{\bigoplus}
\newcommand\srprod{\bigotimes}

\newcommand\inalph{\Sigma}
\newcommand\outalph{\Omega}
\newcommand\states{Q}
\newcommand\transitions{E}
\newcommand\initstate{i}
\newcommand\finalstate{f}
\newcommand\emptystring{\epsilon}
\newcommand\maybe[1]{(#1 \cup \{ \emptystring \})}

\newcommand\state{q}
\newcommand\trans{t}
\newcommand\src{p}
\newcommand\dest{n}
\newcommand\lab{\ell}
\newcommand\inlab{\lab_i}
\newcommand\outlab{\lab_o}
\newcommand\weight{w}

\newcommand\somealph{\Gamma}
\newcommand\someseq{\gamma}
\newcommand\midlab{\lab_m}

\newcommand\transcomp{\circ}

\newcommand\inseq{\sigma}
\newcommand\outseq{\omega}

\newcommand\kleene[1]{{#1}^{\ast}}
\newcommand\inseqs{\kleene{\inalph}}
\newcommand\outseqs{\kleene{\outalph}}
\newcommand\someseqs{\kleene{\somealph}}

\newcommand\tfunc[1]{\mathbb{#1}}

\newcommand\bigo{{\cal O}}

\newcommand\inseqpast{\inseq_p}
\newcommand\inseqnext{\inseq_n}

\begin{document}

\newcommand\authorstring{
Ian Holmes$^{1,\ast}$ \\
\textbf{1} Department of Bioengineering, University of California, Berkeley, CA, USA \\
$\ast$ E-mail: ihh@berkeley.edu
}

\newcommand\titlestring{Transducer codes for DNA}
\newcommand\shorttitlestring{Transducer codes for DNA}
\markboth{\shorttitlestring}{\shorttitlestring}
\begin{flushleft}
{\Large \textbf{\titlestring} } \\
\authorstring
\end{flushleft}
%\section*{Abstract}
%\paragraph{Keywords:}
%\tableofcontents

\section*{Introduction}

Transducers \cite{MohriPereiraRiley2000,WikipediaTransducers}

In bioinformatics:
protein classification \cite{EskinEtAl2000},
phylogenetics \cite{PatenEtAl2008,WestessonEtAlArxiv2012,WestessonEtAl2012},
cancer informatics \cite{SchwarzEtAl2014}

Arithmetic coding \cite{Mackay2003}

DNA storage \cite{ChurchEtAl2012,GoldmanEtAl2013}.
Codes \cite{YazdiEtAl2015}


\section*{Methods}

\subsection*{Weighted finite-state transducers}

Following \cite{MohriPereiraRiley2000}:
Assume a general semiring
$\semiring=(\srset,\srplus,\srtimes,\srplusid,\srtimesid)$
which for our purposes is typically the probability semiring
$(\Re,+,\times,0,1)$
or the tropical semiring
$(\Re_{+} \cup {\infty},\min,+,\infty,0)$.

A weighted finite-state transducer is defined as a tuple
$T = (\inalph,\outalph,\states,\transitions,\initstate,\finalstate)$
consisting of an input alphabet $\inalph$,
an output alphabet $\outalph$ (both alphabets being finite sets),
a finite set of states $\states$,
a finite set of transitions
$\transitions \subseteq \states \times \maybe{\inalph} \times \maybe{\outalph} \times \srset \times \states$,
an initial state $\initstate \in \states$
and a final state $\finalstate \in \states$.

The transducer $T$ can be thought of as an edge-labelled directed graph
where each state is a node
and each transition
$\trans = (\src[\trans],\inlab[\trans],\outlab[\trans],\weight[\trans],\dest[\trans]) \in \transitions$
is an edge from state $\src[\trans]$ to state $\dest[\trans]$
with input label $\inlab[\trans]$,
output label $\outlab[\trans]$
and weight $\weight[\trans]$.

A path in $T$ is a series of transitions that form a path in this graph.
The input sequence and output sequence for a path are the concatenation of (respectively)
the input and output labels of the transitions in the path.
The path weight is the $\srtimes$-product of the transition weights.
A successful path is one that starts in $\initstate$ and ends in $\finalstate$.
The transduction weight for a given input sequence $\inseq \in \inseqs$
and output sequence $\outseq \in \outseqs$
is the $\srplus$-sum of all successful paths
having $\inseq$ as the input sequence
and $\outseq$ as the output sequence.
Thus $T$ provides a mapping
$\tfunc{T}:(\kleene{\inalph} \times \kleene{\outalph}) \to \srset$
from sequence-pairs to weights.
We call this mapping $\tfunc{T}$ the transducer function.
For a given pair of sequences $(\inseq,\outseq)$
and a semiring wherein $\srplus$ and $\srtimes$ are amortized-constant resource operations,
it can be evaluated in time $\bigo(|\inseq| \cdot |\outseq| \cdot |\transitions|)$
and memory $\bigo(|\inseq| \cdot |\outseq| \cdot |\states|)$
by dynamic programming,
analogously to the Forward algorithm in the probabilistic semiring
or the Viterbi algorithm in the tropical semiring
\cite{Durbin98}.

\subsection*{Transducer composition}

Given transducers
 $R = (\inalph, \somealph, \states_R, \transitions_R, \initstate_R, \finalstate_R)$ and
 $S = (\somealph, \outalph, \states_S, \transitions_S, \initstate_S, \finalstate_S)$
which are compatible ($R$'s output alphabet, $\somealph$, is also $S$'s input alphabet),
we can readily find a composite transducer
 $T = R \transcomp S = (\inalph, \outalph, \states_T, \transitions_T, \initstate_T, \finalstate_T)$
such that, if $\tfunc{R}$, $\tfunc{S}$ and $\tfunc{T}$ are the corresponding transducer functions,
then
\[
\forall \inseq \in \inseqs, \outseq \in \outseqs:
\quad
\tfunc{T}(\inseq,\outseq) = \srsum_{\someseq \in \someseqs} \tfunc{R}(\inseq,\someseq) \tfunc{R}(\someseq,\outseq)
\]
that is, $T$ models the feeding of $R$'s output into $S$'s input
(and this intermediate sequence is then summed out---i.e. marginalized, if we are in the probabilistic semiring).

Loosely speaking, we can construct $T$ using the following recipe:
\begin{itemize}
\item Each $T$-state corresponds to a pair of $R$- and $S$-states,
so $\state_T = (\state_R, \state_S)$
and $\states_T \subseteq \states_R \times \states_S$.
\item The initial $T$-state $\initstate_T=(\initstate_R,\initstate_S)$ pairs the initial states of $R$ and $S$.
\item The final $T$-state $\finalstate_T=(\finalstate_R,\finalstate_S)$ pairs the final states of $R$ and $S$.
\item $T$-transitions
$\trans_T = ((\src_R,\src_S),\inlab,\outlab,\weight_T,(\dest_R,\dest_S))$
represent (summaries of sets of) synchronized pairs of $R$-transitions
$\trans_R = (\src_R,\inlab,\midlab,\weight_R,\dest_R)$
and $S$-transitions
$\trans_S = (\src_S,\midlab,\outlab,\weight_S,\dest_S)$
which share the same intermediate label $\midlab$.
The composite transition weight $\weight_T$ is the product $\weight_R \srtimes \weight_S$,
or the sum over such products if there are multiple transition-pairs $(\trans_R,\trans_S)$
consistent with a given $\trans_T$
(which, in the probabilistic semiring, is equivalent to marginalizing out $\midlab$).
\item Some additional manipulation is required to synchronize
transitions involving empty input labels (``insertions''),
empty output labels (``deletions''),
or both (``null transitions'').
For example, we can require that $S$-transitions accepting incoming symbols
all originate from ``ready'' states which have no outgoing insertions.
(If $S$ does not comply with this stipulation then we can easily derive a transducer
with the same transducer function $\tfunc{S}$ that does comply.)
\end{itemize}

Examples of this construction are given in \cite{MohriPereiraRiley2000} and \cite{WestessonEtAlArxiv2012,WestessonEtAl2012}.

\subsection*{De Bruijn graphs}

\subsection*{A transducer for encoding signals in locally non-repetitive DNA}

\subsection*{A transducer that flushes periodically}

\subsection*{A transducer that converts a binary sequence into a mixture of binary, trinary and quaternary}



\bibliography{trans}



\end{document}
